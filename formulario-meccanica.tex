\documentclass[11pt,landscape]{article}

\usepackage{packages}
% Source: Dave Richeson (divisbyzero.com), Dickinson College
% 
% A one-size-fits-all LaTeX cheat sheet. Kept to two pages, so it 
% can be printed (double-sided) on one piece of paper
% 
% Feel free to distribute this example, but please keep the referral
% to divisbyzero.com

\ifthenelse{\lengthtest { \paperwidth = 11in}}
{ \geometry{top=.5in,left=.5in,right=.5in,bottom=.5in} }
{\ifthenelse{ \lengthtest{ \paperwidth = 297mm}}
    {\geometry{top=1cm,left=1cm,right=1cm,bottom=1cm} }
    {\geometry{top=1cm,left=1cm,right=1cm,bottom=1cm} }
}
\pagestyle{empty}
\makeatletter
\renewcommand{\section}{\@startsection{section}{1}{0mm}%
                            {-1ex plus -.5ex minus -.2ex}%
                            {0.5ex plus .2ex}%x
                            {\normalfont\large\bfseries}}
\renewcommand{\subsection}{\@startsection{subsection}{2}{0mm}%
                            {-1explus -.5ex minus -.2ex}%
                            {0.5ex plus .2ex}%
                            {\normalfont\normalsize\bfseries}}
\renewcommand{\subsubsection}{\@startsection{subsubsection}{3}{0mm}%
                            {-1ex plus -.5ex minus -.2ex}%
                            {1ex plus .2ex}%
                            {\normalfont\small\bfseries}}
\makeatother
\setcounter{secnumdepth}{0}
\setlength{\parindent}{0pt}
\setlength{\parskip}{0pt plus 0.5ex}


\begin{document}

\raggedright
\footnotesize

\begin{center}
     \Large{\textbf{Formulario di meccanica analitica}} \\
\end{center}
\begin{multicols}{3}
\setlength{\premulticols}{1pt}
\setlength{\postmulticols}{1pt}
\setlength{\multicolsep}{1pt}
\setlength{\columnsep}{2pt}

% vectors should appear with an underline
\renewcommand{\vec}[1]{\underline{#1}}
% useful command to write versors as "e" with ^ above them
\newcommand{\e}[0]{\hat{e}}
% useful command to write parial derivatives
\newcommand{\de}[2]{\frac{\partial #1}{\partial #2}}
% useful command to write total derivative in time (very common in mechanics)
\newcommand{\ddt}[0]{\frac{\dif}{\dif t}}
% useful command to write the reference frame "I" in nice math italic
\newcommand{\I}[0]{\mathcal{I}}
% useful commando to display the L of the lagrangian and hamiltonian functions in nice italic 
\newcommand{\Lag}[0]{\mathcal{L}}
\newcommand{\Ham}[0]{\mathcal{H}}

\newcommand{\dipdue}[0]{(t, q^k)}
\newcommand{\diptre}[0]{(t, q^k, \dot q ^k)}

\section{Trasformazioni di coordinate}
    \subsection{Trasformazione generica}
        Consideriamo una trasformazione di coordinate $T$ tale che 
        $$ \begin{pmatrix}
            x\\
            y\\
            z\\
        \end{pmatrix}
        = 
        \begin{pmatrix}
            x(u,v,w)\\
            y(u,v,w)\\
            z(u,v,w)\\
        \end{pmatrix}
        = T \begin{pmatrix}
            u\\ v\\ w\\
            \end{pmatrix}
        $$
        allora per trovare ad esempio il versore $\e_u$ fissi $v$ e $w$ e ottieni una curva al variare del parametro $u$. Allora 
        $$ \e_u = \frac{\vec{t}_u}{\norm{\vec{t}_u}} \quad \text{con} \quad \vec{t}_u = \frac{\partial T}{\partial u} \cdot \begin{pmatrix} \hat{e}_x\\ \hat{e}_y \\ \hat{e}_z \end{pmatrix} = \de{x}{u} \e_x + \de{y}{u} \e_y + \de{z}{u} \e_z$$
        In sostanza puoi scrivere lo Jacobiano $J$ della trasformazione (derivate rispetto alla stessa variabile sulla stessa riga) e ottenere
        $$ \text{da normalizzare}\longrightarrow \begin{pmatrix}
            \e_u\\ \e_v\\ \e_w
        \end{pmatrix} = J^T
        \begin{pmatrix}
            \e_x \\\e_y\\\e_z
        \end{pmatrix}$$
    \subsection{Coordinate cilindriche}
        Siano $r$ distanza all'origine della proiezione del punto sul piano $xy$; $\theta \in [0, 2\pi)$ angolo tra $r$ e l'asse $x$; $z$ solita $z$.
        $$ \begin{cases}
            x = r \cos \theta \\
            y = r \sin \theta \\
            z = z
        \end{cases}$$
        $$\begin{pmatrix} \e_r \\ \e_\theta \\ \e_z \end{pmatrix} = 
            \begin{pmatrix}
                \cos \theta & \sin \theta & 0 \\
                -\sin \theta & \cos \theta & 0 \\
                0 & 0 & 1
            \end{pmatrix}
        \begin{pmatrix} \e_x \\\e_y \\\e_z \end{pmatrix}$$
        Derivazione dei versori, velocità e accelerazione in coordinate cilindriche (rispetto a sist. rif. inerziale):
        $$ \begin{cases}
            \dot{\e}_r = \dot{\theta} \e_\theta\\
            \dot{\e}_\theta = -\dot{\theta} \e_r\\
            \dot{\e}_z = 0
        \end{cases}$$
        $$ \vec{x}_P = r\e_r + z\e_z$$
        $$ \vec{v}_P = \dot{r} \e_r + r \dot{\theta} \e_\theta + \dot{z}\e_z$$
        $$ \vec{a}_P = (\ddot{r} - r \dot{\theta}^2)\e_r + (2 \dot{\theta}\dot{r} + r\ddot{\theta})\e_\theta + \ddot{z} \e_z$$
    \subsection{Coordinate sferiche}
        Siano $r$ distanza dall'origine; $\theta \in [0,\pi]$ angolo tra $r$ e l'asse $z$; $\phi \in [0, 2\pi)$ angolo tra l'asse $x$ e la proiezione di $r$ sul piano $xy$, ovvero $r \sin \theta$.
        $$\begin{cases}
            x = r \sin \theta \cos \phi \\
            y = r \sin \theta \sin \phi \\
            z = r \cos \theta
        \end{cases}$$
        $$\begin{pmatrix} \e_r \\ \e_\theta \\ \e_\phi \end{pmatrix} = 
        \begin{pmatrix}
            \sin \theta \cos \phi & \sin \theta \sin \phi & \cos \theta \\
            \cos \theta \cos \phi & \cos \theta \sin \phi & -\sin \theta \\
            - \sin \phi & \cos \phi & 0
        \end{pmatrix}
        \begin{pmatrix} \e_x \\\e_y \\\e_z \end{pmatrix}$$
        Derivazione dei versori, posizione, velocità e accelerazione in coordinate sferiche:
            $$\begin{cases}
                \dot{\e}_r = \dot{\theta}\e_\theta + \dot{\phi}\sin\theta \e_\phi\\
                \dot{\e}_\theta = -\dot{\theta}\e_r + \dot{\phi}\cos \theta \e_\phi\\
                \dot{\e}_\phi = -\dot{\phi}\sin \theta \e_r - \dot{\phi}\cos \theta \e_\theta
            \end{cases}
            $$
            $$ \vec{x}_P = r\e_r$$
            $$ \vec{v}_P = \dot{r} \e_r + r\dot{\theta}\e_\theta + r\dot{\phi}\sin \theta \e_\phi$$
            $$ \vec{a}_P = (\ddot{r} - r\dot{\phi} - r\dot{\phi}^2 \sin^2 \theta )\e_r + (r\ddot{\theta} + 2\dot{r}\dot{\theta} - r\dot{\phi}^2 \cos \theta \sin \theta ) \e_\theta +$$ $$ + (r\sin\theta \ddot{\phi} + 2r \dot{\phi} \dot{\theta} \cos \theta  + 2 \dot{r} \dot{\phi} \sin \theta)\e_\phi $$
    \subsection{Coordinate toroidali}
        Punto vincolato a muoversi su una superficie toroidale: sia $R$ la distanza (fissa) dal centro del toro al centro della sua sezione circolare, $r$ il raggio (fisso) della sezione del toro, $\theta \in [0, \pi]$ angolo tra $r$ e la verticale, $\phi \in [0, 2\pi)$ angolo tra $R$ e l'asse $x$. 
        $$ \begin{cases}
            x = (R + r \sin \theta) \cos \phi\\
            y = (R + r \sin \theta) \sin \phi\\
            z = r \cos \theta
        \end{cases}$$ 
        $$ \begin{pmatrix}
            \e_r\\ \e_\theta \\ \e_\phi
        \end{pmatrix} = 
        \begin{pmatrix}
            \sin \theta \cos \phi & \sin \theta \sin \phi & \cos \theta \\
            \cos \theta \cos \phi & \cos \theta \sin \phi & -\sin \theta \\
            - \sin \phi & \cos \phi & 0
        \end{pmatrix} \begin{pmatrix} \e_x \\ \e_y \\ \e_z \end{pmatrix}$$
        Derivazione dei versori, posizione, velocità e accelerazione:
        $$ \begin{cases}
            \dot{\e}_r = \dot{\theta} \e_\theta + \dot{\phi} \sin \theta \e_\phi\\
            \dot{\e}_\theta = -\dot{\theta} \e_r + \dot{\phi} \cos \theta \e_\phi\\
            \dot{\e}_\phi  = - \dot{\phi} \sin \theta \e_r - \dot{\phi}\cos \theta \e_\theta
        \end{cases}$$
        $$ \vec{x}_P = (r + R\sin \theta)\e_r + R\cos \theta \e_\theta$$
        $$ \vec{v}_P = r \dot{\theta} \e_\theta + (R + r\sin \theta) \dot{\phi} \e_\phi$$
        $$ \vec{a}_P =\big[-r \dot \theta ^2 - (R + r\sin \theta) \dot \phi ^2 \sin \theta \big] \e_r +$$ $$ + \big[r \ddot \theta - (R + r\sin \theta) \dot \phi ^2 \cos \theta \big]\e_\theta + $$ $$ + \big[2r \dot \theta \dot \phi \cos \theta + (R + r\sin \theta) \ddot \phi \big]\e_\phi$$
    
\section{Velocità angolare e formula di Poisson}
        Siano $\I$ e $\I'$ sist. rif. con basi ad essi solidali $\{\e_i\}$ e $\{\e_i'\}$
    $$\vec{\omega}_{\I/\I'} = \frac 1 2 \sum_i \e_i \wedge \ddt_{\I'} \e_i'$$
    $$ \ddt_\I \e_i' = \vec{\omega} \wedge \e_i$$
    In cooridinate sferiche $$ \vec{\omega} = \dot \phi \cos \theta \e_r - \dot \phi \sin \theta \e_\theta + \dot \theta \e_\phi $$

\section{Corpi rigidi}
    \subsection{Momento della quantità di moto}
    Sia $\beta$ corpo rigido e $A(t)$ punto generico.
    $$ \vec{\mathcal{L}}_{A(t)\rvert{\I}}(t) := \int \limits_{P\in \beta} \big (P(t) - A(t) \big) \wedge {\vec{v}_P}_{\rvert \I}\dif m$$
    \subsection{Operatore di inerzia}
    Si definisce l'operatore di inerzia rispetto ad $A(t)$ del corpo $\beta$:
        $$ I_{A(t)} : V_t \longrightarrow V_t$$
        $$ \vec{I}_{A(t)}(\vec w) = \int \limits_{P \in \beta} \big( P- A\big) \wedge \big( \vec w \wedge (P-A)\big) \dif m$$
    E allora risulta, detto $G$ centro di massa del corpo \\
    - se $A$ coincide con un punto materiale di $\beta$:
        $$ \vec{\mathcal{L}}_{A(t)\rvert{\I}}(t) = M_\beta \big(G - A \big) \wedge {\vec{v}_A}_{\rvert_\I} + \vec{I}_{A(t)}(\vec \omega _{\beta/\I})$$
    - se $A$ è generico:
        $$ \vec{\mathcal{L}}_{A(t)\rvert{\I}}(t) = M_\beta \big(G - A \big) \wedge {\vec{v}_G}_{\rvert_\I} + \vec{I}_{G(t)}(\vec \omega _{\beta/\I})$$
    \subsection{Momento d'inerzia}
    Si definisce momento d'inerzia del corpo rispetto a un asse di direzione $\hat u$ passante per $A$ lo scalare 
    $$ I_{A, \hat u} := \hat u \cdot \vec{I}_A(\hat u)$$
    \subsection{Matrice d'inerzia}
    La matrice relativa all'operatore di inerzia è comoda se calcolata rispetto a un polo e una base $\{\e_i'\}$ solidali al corpo rigido, perché così non dipende più dal tempo, e risulta:
        $$\{I'_A\}_{ij} := \e_i' \cdot \vec{I}_A(\e_j') = $$ 
        $$ = \int \limits_{P \in \beta} \delta_{ij} \norm{P-A}^2 - (P-A)_i (P-A)_j \dif m$$
    \subsection{Teorema di Steiner}
    Sia $A$ punto solidale al corpo rigido. Data sempre una base solidale al corpo rigido, vale 
        $$ \{I_A\}_{ij} = \{I_G\}_{ij} + M_\beta \bigg (\delta_{ij} \norm{G-A}^2 - (G-A)_i (G-A)_j \bigg)$$
    \subsection{Energia cinetica di un corpo rigido}
        $$T_{\beta/\I} = \frac 1 2 M_\beta \norm{{\vec v _G}_{\rvert_\I}}^2 + \frac 1 2 {\vec \omega}_{\beta/\I} \cdot \vec{I}_G(\vec{\omega}_{\beta/\I})$$
    \subsection{II equazione cardinale del moto}
        $$ {\ddt}_{\rvert_\I} {\vec{\mathcal{L}}_{A}}_{\rvert_\I} + m_\beta {\vec{v}_A}_{\rvert_\I} \wedge {\vec{v}_G}_{\rvert_A} = {\vec{M}^{ext}_A}_{\rvert_\I}$$

\section{Meccanica Lagrangiana}
    Abbiamo $N$ punti materiali $P_1,\dots, P_N$ e $n$ coordinate lagrangiane $q^1, \dots, q^n$. Da ora in poi è sott'inteso il sistema di riferimento.
    \subsection{Condizione di idealità dei vincoli}
        Detta $\vec \phi _i$ la risultante dei vincoli agenti sul punto $P_i$:
        $$ \sum_{i=1}^N \vec{\phi}_i \cdot \de{\vec{x}_{P_i}}{q^k} = 0$$
    \subsection{Equazioni del moto di Eulero-Lagrange}
        Per $k = 1, \dots, n$
        $$\ddt \bigg( \de{T}{\dot q ^k}\diptre \bigg)- \de{T}{q^k}\diptre = Q_k (t, q^k , \dot q^k)$$
        $$ \text{con} \quad Q_k \diptre = \sum_{i=1}^N \vec F_i\cdot \de{\vec x_{P_i}}{q^k}$$
        dove $\vec F_i$ è la risultate delle forze attive agenti sul punto $P_i$ (comprese quelle apparenti, se ci sono).
    \subsection{Identità}
        $$\de{\vec x_{P_i}}{q^k} = \de{\vec v_{P_i}}{\dot q^k}$$
        $$ \ddt \bigg(\de{\vec x_{P_i}}{q^k}\bigg) = \de{\vec v_{P_i}}{q^k}$$
    \subsection{Potenziale}
        $U \diptre$ tale che $$ Q_k \diptre = \de{U}{q^k} - \ddt\de{U}{\dot q ^k} \quad k = 1,\dots,n$$
        Condizioni necessarie: per ogni $k,r = 1,\dots, n$
        $$ \de{Q_k}{\dot q^r} + \de{Q_r}{\dot q^k} = 0$$
        $$\de{Q_k}{q^r} - \de{Q_r}{q^k} = \ddt \de{Q_k}{\dot q^r}$$
        Il potenziale può dipendere solo \underline{linearmente} dalle $\dot q ^k$:
        $$ U \diptre = U_0 \dipdue + \sum_{k=1}^n U_k \dipdue \dot q^k$$
        e inoltre se le $Q_k$ non dipendono da una certa variabile, puoi provare a omettere la dipendenza da essa nel potenziale e vedere se tutto funziona.
        \subsubsection{Potenziale delle forze apparenti}
        Sia $\I$ sistema di riferiemento inerziale e $\I'$ generico. Nel sistema $\I'$ il potenziale dovuto alle forze apparenti vale
        $$ U^{app} = T_\I - T_{\I'}$$
    \subsection{Funzione lagrangiana}
        $$ \Lag \diptre := T + U$$
        $$ \ddt \de{\Lag}{\dot q^k} - \de{\Lag}{q^k} = 0$$
    \subsection{Hamiltoniana}
        $$ \Ham := \bigg(\sum_{k=1}^n \de{\Lag}{\dot q^k} \dot q^k \bigg) - \Lag = $$
        $$ = T_2 - T_0 - U_0$$
        Se $\Lag$ non dipende esplicitamente dal tempo, $\Ham$ si conserva lungo ogni moto.
\section{Meccanica Hamiltoniana}
    \subsection{Trasformazione di Legendre}
        $$ \begin{cases}
            t = t\\
            q^k = q^k\\
            p_k = \de{\Lag}{\dot q^k}
        \end{cases}$$
    \subsection{Equazioni di Hamilton}
        $$ \begin{cases}
            \frac{\dif t}{\dif t} = 1\\
            \frac{\dif q^k}{\dif t} = \de{\Ham}{p_k}\\
            \frac{\dif p_k}{\dif t} = - \de{\Ham}{q^k}
        \end{cases}$$
    \subsection{Forma differenziale di Poincaré-Cartan}
        Il campo dinamico $\mathcal{Z}$ diventa 
        $$ \mathcal{Z} = \de{}{t} + \sum_k \de{\Ham}{p_k} \de{}{q^k} - \de{\Ham}{q^k} \de{}{p_k}$$
        Ed è univocamente determinato dalla forma differenziale di Poincaré-Cartan
        $$ \dif \theta_H = -\Ham \dif t + \sum_i p_i \dif q^i$$
    \subsection{Condizione di Lie}
        Dico che un cambiamento di coordinate del tipo 
        $$ \begin{cases}
            t' = t\\
            q'^k = q'^k (t, q^s, p_s)\\
            p'_s = p'_s(t, q^s, p_s)
        \end{cases}$$
        soddisfa la condizione di Lie se nelle nuove coordinate
        $$ \dif \theta_{\Ham'} = - \Ham' \dif t + \sum_i p'_i \dif q'^i + \dif F$$
        e ciò implica che la struttura Hamiltoniana di $\mathcal{Z}$ è conservata. 
        Se vale la condizione di Lie, allora ipotizzando $F = F(t, q^k, q'k)$
        $$ \begin{cases}
            \Ham' = \Ham + \de{F}{t}\\
            p_i = \de{F}{q^i}\\
            p'_i = - \de{F}{q'^i}
        \end{cases}$$
    \subsection{Equazione di Hamilton-Jacobi}
        Vorrei trovare una trasformazione per cui $\Ham' = 0$ sempre, e quindi
        $$ \Ham(t, q^k, \de{F}{q^k}) + \de{F}{t} = 0$$
\end{multicols}
\end{document}