\documentclass[11pt,landscape]{article}

\usepackage{packages}
% Source: Dave Richeson (divisbyzero.com), Dickinson College
% 
% A one-size-fits-all LaTeX cheat sheet. Kept to two pages, so it 
% can be printed (double-sided) on one piece of paper
% 
% Feel free to distribute this example, but please keep the referral
% to divisbyzero.com

\ifthenelse{\lengthtest { \paperwidth = 11in}}
{ \geometry{top=.5in,left=.5in,right=.5in,bottom=.5in} }
{\ifthenelse{ \lengthtest{ \paperwidth = 297mm}}
    {\geometry{top=1cm,left=1cm,right=1cm,bottom=1cm} }
    {\geometry{top=1cm,left=1cm,right=1cm,bottom=1cm} }
}
\pagestyle{empty}
\makeatletter
\renewcommand{\section}{\@startsection{section}{1}{0mm}%
                            {-1ex plus -.5ex minus -.2ex}%
                            {0.5ex plus .2ex}%x
                            {\normalfont\large\bfseries}}
\renewcommand{\subsection}{\@startsection{subsection}{2}{0mm}%
                            {-1explus -.5ex minus -.2ex}%
                            {0.5ex plus .2ex}%
                            {\normalfont\normalsize\bfseries}}
\renewcommand{\subsubsection}{\@startsection{subsubsection}{3}{0mm}%
                            {-1ex plus -.5ex minus -.2ex}%
                            {1ex plus .2ex}%
                            {\normalfont\small\bfseries}}
\makeatother
\setcounter{secnumdepth}{0}
\setlength{\parindent}{0pt}
\setlength{\parskip}{0pt plus 0.5ex}


\begin{document}

\raggedright
\footnotesize

\begin{center}
     \Large{\textbf{Formulario di meccanica analitica}} \\
\end{center}
\begin{multicols}{3}
\setlength{\premulticols}{1pt}
\setlength{\postmulticols}{1pt}
\setlength{\multicolsep}{1pt}
\setlength{\columnsep}{2pt}

% vectors should appear with an underline
\renewcommand{\vec}[1]{\underline{#1}}
% useful command to write versors as "e" with ^ above them
\newcommand{\e}[0]{\hat{e}}
% useful command to write parial derivatives
\newcommand{\de}[2]{\frac{\partial #1}{\partial #2}}

\section{Trasformazioni di coordinate}
    \subsection{Trasformazione generica}
        Consideriamo una trasformazione di coordinate $T$ tale che 
        $$ \begin{pmatrix}
            x\\
            y\\
            z\\
        \end{pmatrix}
        = 
        \begin{pmatrix}
            x(u,v,w)\\
            y(u,v,w)\\
            z(u,v,w)\\
        \end{pmatrix}
        = T \begin{pmatrix}
            u\\ v\\ w\\
            \end{pmatrix}
        $$
        allora per trovare ad esempio il versore $\e_u$ fissi $v$ e $w$ e ottieni una curva al variare del parametro $u$. Allora 
        $$ \e_u = \frac{\vec{t}_u}{\norm{\vec{t}_u}} \quad \text{con} \quad \vec{t}_u = \frac{\partial T}{\partial u} \cdot \begin{pmatrix} \hat{e}_x\\ \hat{e}_y \\ \hat{e}_z \end{pmatrix} = \de{x}{u} \e_x + \de{y}{u} \e_y + \de{z}{u} \e_z$$
        In sostanza puoi scrivere lo Jacobiano $J$ della trasformazione (derivate rispetto alla stessa variabile sulla stessa riga) e ottenere
        $$ \text{da normalizzare}\longrightarrow \begin{pmatrix}
            \e_u\\ \e_v\\ \e_w
        \end{pmatrix} = J^T
        \begin{pmatrix}
            \e_x \\\e_y\\\e_z
        \end{pmatrix}$$
    \subsection{Coordinate cilindriche}
        Siano $r$ distanza all'origine della proiezione del punto sul piano $xy$; $\theta \in [0, 2\pi)$ angolo tra $r$ e l'asse $x$; $z$ solita $z$.
        $$ \begin{cases}
            x = r \cos \theta \\
            y = r \sin \theta \\
            z = z
        \end{cases}$$
        $$\begin{pmatrix} \e_r \\ \e_\theta \\ \e_z \end{pmatrix} = 
            \begin{pmatrix}
                \cos \theta & \sin \theta & 0 \\
                -\sin \theta & \cos \theta & 0 \\
                0 & 0 & 1
            \end{pmatrix}
        \begin{pmatrix} \e_x \\\e_y \\\e_z \end{pmatrix}$$
    \subsection{Coordinate sferiche}
        Siano $r$ distanza dall'origine; $\theta \in [0,\pi]$ angolo tra $r$ e l'asse $z$; $\phi \in [0, 2\pi)$ angolo tra l'asse $x$ e la proiezione di $r$ sul piano $xy$, ovvero $r \sin \theta$.
        $$\begin{cases}
            x = r \sin \theta \cos \phi \\
            y = r \sin \theta \sin \phi \\
            z = r \cos \theta
        \end{cases}$$
        $$\begin{pmatrix} \e_r \\ \e_\theta \\ \e_\phi \end{pmatrix} = 
        \begin{pmatrix}
            \sin \theta \cos \phi & \sin \theta \sin \phi & \cos \theta \\
            \cos \theta \cos \phi & \cos \theta \sin \phi & -\sin \theta \\
            - \sin \phi & \cos \phi & 0
        \end{pmatrix}
        \begin{pmatrix} \e_x \\\e_y \\\e_z \end{pmatrix}$$
    \subsection{Coordinate toroidali}

    


\end{multicols}
\end{document}